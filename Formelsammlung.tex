\documentclass[twoside,10pt,a5paper]{scrartcl}

\usepackage[ngerman]{babel}
\usepackage[bottom=2cm, left=1cm, right=1cm, top=1cm]{geometry}
\usepackage{booktabs}

\setlength{\parindent}{0pt}
\setlength{\parskip}{1ex}

\begin{document}
\sffamily

\vspace*{1cm}

{\centering{\usekomafont{disposition}
    \LARGE Digitale Signalverarbeitung --- Klausurhilfe}\\[1ex]
  Tim Hilt\\
  \today
  \par}

\vspace{2cm}

\tableofcontents
\clearpage

\section{Umformung Impulsantwort $\mathbf{\Leftrightarrow}$ Sprungantwort}

\[H(z) = A(z) \cdot (1 - z^{-1}) \qquad\Rightarrow\qquad A(z) = \frac{H(z)}{1 - z^{-1}} = \frac{zH(z)}{z-1}\]

\[a_{kT} = a_{kT - T} + h_{kT} \qquad\Rightarrow\qquad h_{kT} = a_{kT} - a_{kT - T}\]

\section{Differenzengleichung aus System erstellen}

\begin{enumerate}
\item Einführung von Variablen an den Ausgängen sämtlicher Addierer
\item Aufstellen von Gleichungen an den Ausgängen aller Addierer
\end{enumerate}

\section{Koeffizienten der $z$-Übertragungsfunktion}

\[\tilde{H}(z) = \frac{L_0 \cdot z^N + L_1 \cdot z^{N-1} + L_2 \cdot z^{N-2} + \ldots+ L_N}{z^N - K_1 \cdot z^{N-1} - K_2\cdot z^{N-2} - \ldots - K_N}\]

\section{Stabilität eines zeitdiskreten Systems}

Ein zeitdiskretes System ist dann stabil, wenn die Pole der \(z\)-transformierten Übertragungsfunktion in der \(z\)-Ebene innerhalb oder auf dem Einheitskreis liegen; d.h.\ wenn gilt:

\[|z_{pi}| \leq 1, \qquad i \in 1, 2, \ldots , N\]

\section{Impulsantwort des Systems}

Ist nach der Impulsantwort des Systems gefragt, so kann der Ausgang des Systems \(h(kT) = y(kT)\) verwendet werden.

\section{Amplitudengang $H(f)$ aus $H(z)$ berechnen}

Ist die \(z\)-Übertragungsfunktion \(\tilde{H}(z)\) bekannt und der Amplitudengang gesucht, kann beachtet werden, dass die Frequenzen von \(f = 0\) bis \(f = f_T/2\) in der \(z\)-Ebene abgelesen werden können, wenn auf dem Einheitskreis von \(1\) nach \(-1\) gegangen wird. Demnach wäre \(f_T/4\) bei \(j\).

\section{Sprungantwort $a(kT)$ mit $k \rightarrow \infty$}

Ist der Konvergenzwert / stationäre Endwert der Sprungantwort \(a(kT)\) eines Systems gesucht, so kann in die \(z\)-Übertragungsfunktion \(\tilde{H}(z)\) für \(z = 1\) eingesetzt werden, um den richtigen Wert zu erhalten.

\[\lim_{k \rightarrow \infty}a(kT) = \sum_{i=0}^k h(kT) = \tilde{H}(z = 1)\]

\section{Realisierbarkeit von Systemen}

Für eine realisierbare Funktion muss gelten:

\begin{enumerate}
\item Die Übertragungsfunktion muss eine gebrochen-rationale Funktion sein (\(\frac{z^2+z}{z^3+z+1}\))
\item Die Übertragungsfunktion muss reelle Koeffizienten haben
\item Der Zählergrad muss kleiner oder gleich dem Nennergrad sein
\end{enumerate}

\section{FIR (Finite-Impulse-Response)-Filter}

\subsection{Filterklassen}

\begin{center}
  \begin{tabular}{ccc}
    \toprule
    & Achsensymmetrie & Punktsymmetrie\\
    \midrule
    gerader Grad \(N\) & Klasse 1 & Klasse 2\\
    ungerader Grad \(N\) & Klasse 3 & Klasse 4\\
    \bottomrule
  \end{tabular}
\end{center}

Der Grad kann ermittelt werden, indem in der Impulsantwort \(h(kT)\) die \textbf{Anzahl der Verzögerungen} gezählt wird.

\end{document}

%%% Local Variables:
%%% mode: latex
%%% TeX-master: t
%%% End: